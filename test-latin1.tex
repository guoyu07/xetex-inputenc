%
% TESTING XETEX-INPUTENC
% ======================
%
% Three possibilities for compiling this document:
%
% 1.   pdflatex test-latin1
% 2.   xelatex  test-latin1
% 3.   xelatex  "\let\iffontspec\iftrue\input test-latin1"
%
% Compilations #1 and #2 both compile the Latin 1--encoded document
% with a legacy .tfm-based font in T1 encoding.
%
% Compilation #3 compiles the document with a unicode font.
%
% All three should produce identical output.
% 

\ifcsname iffontspec\endcsname
  \def\next#1{}
\else
  \def\next#1{#1}
\fi
\next{\let\iffontspec\iffalse}

\documentclass{article}
\usepackage{ifxetex}
\ifxetex
  \usepackage[latin1]{xetex-inputenc}
\else
  \usepackage[latin1]{inputenc}
\fi
\iftrue
  \usepackage[T1]{fontenc}
  \usepackage{lmodern}
\else
  \usepackage{fontspec}
\fi

\begin{document}

\tableofcontents
\listoffigures

\section{Main file: � � � � �}
\subsection{\protect � \protect � \protect � \protect � \protect �}
\subsection{\string � \string � \string � \string � \string �}

\input test-latin1-subfile.tex

\begin{figure}[b]
\centering\rule{1em}{1em}
\caption{Caption: � \protect � \string � \unexpanded{�} � .}
\end{figure}

\end{document}